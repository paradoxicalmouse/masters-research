\documentclass{article}
\author{Taylor B. Morris}
\title{Analysis of Stack Overflow}
\date{\today}
\begin{document}
\maketitle
\begin{abstract}
\end{abstract}
\section{Introduction}


\section{Background}
The average programmer today is familiar with Stack Overflow, a forum
designed for asking and answering programming related questions. Stack
Overflow is open to anyone whether they be professional or a hobbyist or
just starting out. Due to its open nature, posts on Stack Overflow cover
a large variety of questions over a number of languages. A quick internet
search for a programming issue is likely to result in a Stack Overflow post
as a top result.

Thanks to the ubiquitous nature of Stack Overflow, analysis on the site
offers a way to obtain a large amount of information on programmers'
practices and problems. One prior analysis of Stack Overflow utilized an 
alternate method of parsing and tagging of data, using an adverb-verb 
formulation instead of the traditional noun and verb word bagging. This 
method of analysis was performed on the topics of Stack Overflow posts and 
allowed the researchers to evaluate the types of questions seen 
corresponding to different programming languages and different Stack 
Overflow tags. This analysis shed light on what types of questions are more 
common between different languages; additionally, this study helped to 
reveal some of the most common uses of each langauge 
\cite{allamanis2013and}.

\section{Methods}
\subsection{Question Data}
The data to be used is a dataset of Stack Overflow questions and some of
their statistics found on Kaggle.com. Each row of the data represents a
single question and includes the title, body, score (a metric used by Stack
Overflow to determine the quality of a question), closed date (if the
question was closed), and owner ID. Additionally, separate tables include
answer and tag data for each question.
The answer table contains each of the answers for each question and their
data. Each answer contains the answer body, creation data, score, and owner
ID. Finally, the tags table contains all of the tags for each question. 

Each Stack Overflow question typically contains a single sentence title. 
Often, a simplified version of the question appears in the title; however,
the primary text and background for the coding question appears in the body.
The body can contain Markdown or HTML code, including a special tag to allow
for code to be included within the question body: {\tt <code>}. 

In order to predict whether a question will be closed, answered, or left
open and unanswered, we need to be able to gather the information of whether
or not the training questions have been answered. This is problematic, as
the initial form of the dataset has each answer as its own row in the
Answers table. While these answers tell their parentID, i.e. the question
they are given in response to, the entire table must be navigated to search
for at least one occurence of our question's ID in the Answers table.

In the cases where the question is closed, this step does not apply, as we
would categorize the question as "closed", and thus not answered. However,
in the other cases, this process can be rather time consuming. To speed up
this process during initial creation, we can shortcut out of the search when
we find an answer to a particular question and mark the question as
answered.

\section{Results}

\section{Conclusions}
\bibliographystyle{plain}
\bibliography{tbm-masters-research} 
\end{document}
